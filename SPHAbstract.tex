\documentclass{article}
\usepackage[utf8]{inputenc}

\title{\textbf{Lagrangian Fluid Simulation}}
\author{Thatcher Christeaan (cs184-bi)\\
       Elliott Ison (cs184-ag)\\
       Kimberly Ng (cs184-et)\\
       Harrison Wang (cs184-cd)\\}
\date{December 12 2013}

\begin{document}

\maketitle

\section{Abstract}
 In this project, we explore the method of Smoothed Particle Hydrodynamics using Lagrangian method of representing flow of the particles in simulating water and other interesting types of liquids such as those of high viscosity (mucus). In our simulations, we display scenarios where water is falling, water and mucus are mixed together, and also a "Dam" scenario to simulate waves. Also, we allowed user input and keyboard interaction that allows useres to place new particles and also change some of the boundaries. In addition, the program can take in an $.obj$ file and take the points in it and create particles out of those. For example, we can simulate a falling teapot made of water. Through SPH, we approximate values through the use of multiple kernel functions depending on what we're evaluating and computation is easier because evaluating gradients only requires taking the gradient of the kernel function. To capture our simulation, we implemented a system to take the OpenGL buffer and write it to an AVI file. Through this, we are able to speed up the speed up of how the simulation looks. We also utilize raytracing for surface reconstruction.\\
 
  In order to speed up computation time, We created a grid with hashing. This speedup is due to a faster "nearest neighbor search" and that instead of looping through all the particles, we utilize a hash function to generate hash keys for each cell which contains a certian amount of particles. We can then just loop through neighbor particles and only consider those for the contribution of pressure and density for each individual particle. We also took advantage of OpenMP parallelism to speed up our simulation.\\
  
  In order to deal with incompressability of the fluids, we simulated
  everything with a time step of 0.005s. Otherwise, it would be unstable and seem as though the particles are under a vortex. Through the Navier-Stokes equation, we were able to determine the particle's accelerations. Then, through leap frog integration, we determine the new positions and velocity of each particle. 
\end{document}

